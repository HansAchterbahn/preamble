% Dokument / Seiten
\documentclass[12pt,a4paper,parskip]{scrartcl}
\usepackage[utf8]{inputenc}				% Zeichencodierung
\usepackage[ngerman]{babel}				% Sprachpaket 
\usepackage[T1]{fontenc}				% Umlaute bei \hyphenation verwenden können (Silbentrennung)
\usepackage[onehalfspacing]{setspace}	% Setzt das Dokument auf 1,5x Zeilenabstand

\usepackage{							% standard Krempel
	amsmath,
	amsfonts,
	amsthm,
	amssymb}

\setlength{\headheight}{1.1\baselineskip}

% Gleitumgebungen / Grafiken
\usepackage{float}						% notwendig für Gleitubgebungen z.B. Figure
\usepackage{subfig}						% erlaubt das Verwenden von Subfigures
\usepackage{graphicx}					% erlaubt das Einbinden von Grafiken
\usepackage{pdfpages}					% ermöglicht das Einbinden von PDFs
\usepackage{epstopdf}					% einbinden von eps Bilddateien
\usepackage{longtable}					% lange Tabellen verarbeiten
\usepackage{caption}					% ermöglicht die Beschriftung von Gleitobjekten
\usepackage{nameref}					% erlaubt Namensreferenzen
\usepackage{wrapfig}                    % Text umschließt Figure

% Diagramme
\usepackage{tikz}						% Zeichentool
\usepackage{pgfplots, pgfplotstable}	% Diagramme zeichnen
\pgfplotsset{compat = newest}			% Einstellung für Diagramme
\usepackage{siunitx}                    % SI-Einheiten
\usepgfplotslibrary{units}				% Einheiten in Diagrammen plotten

% Verzeichnisse
\usepackage[square,authoryear]{natbib}	% Literatur zitieren
\bibliographystyle{dinat}				% Literatur- und Zitatstyle

\usepackage{makeidx}					% anlegen eines automatischen Inhaltsverzeichnis
\usepackage{tocstyle}					% Punkte im Inhaltsverzeichnis
\usetocstyle{allwithdot} 

% Formatierungen
\usepackage{eurosym}					% €-Zeichen mit \euro darstellbar oder durch \EUR{betrag}

\setlength{\parindent}{0in} 			% Absatz, erste Zeile wird global NICHT eingerückt
%\usepackage{parskip}					% setzt einen Abstand nach einem \par Absatz
\usepackage{rotating}
\newcommand\tabrotate[1]{\begin{turn}{90}\rlap{#1}\end{turn}}

\usepackage[headsepline]{scrpage2}		% Kopfzeile

\usepackage[							% Verlinkung aller \ref,url,etc.
colorlinks=true,
urlcolor=blue,
citecolor=blue,
linkcolor=black
]{hyperref}								

\usepackage{wasysym}					% zusätzliche Symbole verwenden (Checkbox)
\usepackage{listings}


% Silbentrennung
\expandafter\def\expandafter\UrlBreaks\expandafter{\UrlBreaks\do\a		% Erlaubt einen Umbruch in einer URL 
	\do\b\do\c\do\d\do\e\do\f\do\g\do\h\do\i\do\j\do\k\do\l\do\m\do\n	% an allen angegebenen Stellen
	\do\o\do\p\do\q\do\r\do\s\do\t\do\u\do\v\do\w\do\x\do\y\do\z\do\A	
	\do\B\do\C\do\D\do\E\do\F\do\G\do\H\do\I\do\J\do\K\do\L\do\M\do\N	
	\do\O\do\P\do\Q\do\R\do\S\do\T\do\U\do\V\do\W\do\X\do\Y\do\Z\do\1
	\do\2\do\3\do\4\do\5\do\6\do\7\do\8\do\9\do\0\do\&} 	 

%\sloppy								% Nachlässige Silbentrennung, Blocksatzoptimiert
\hyphenation{OpenOCD}					% Worte, die NICHT oder nur auf bestimmte Weise getrennt werden sollen

% Schusterjungen und Hurenkinder
\usepackage[all]{nowidow}
%\clubpenalty = 10000 % schliesst Schusterjungen aus
%\widowpenalty = 10000 
%\displaywidowpenalty = 10000% schliesst Hurenkinder aus 

%Include Python-Programmcode with Syntax-Highlighting
% Default fixed font does not support bold face
\DeclareFixedFont{\ttb}{T1}{txtt}{bx}{n}{11} % for bold
\DeclareFixedFont{\ttm}{T1}{txtt}{m}{n}{11}  % for normal

% Custom colors
\usepackage{color}
\definecolor{deepblue}{rgb}{0,0,0.5}
\definecolor{deepred}{rgb}{0.6,0,0}
\definecolor{deepgreen}{rgb}{0,0.5,0}

\usepackage{listings}

% Python style for highlighting
\newcommand\pythonstyle{\lstset{
		language=Python,
		basicstyle=\ttm,
		otherkeywords={self},             % Add keywords here
		keywordstyle=\ttb\color{deepblue},
		emph={MyClass,__init__},          % Custom highlighting
		emphstyle=\ttb\color{deepred},    % Custom highlighting style
		stringstyle=\color{deepgreen},
		frame=tb,                         % Any extra options here
		showstringspaces=false            % 
}}


% Python environment
\lstnewenvironment{python}[1][]
{
	\pythonstyle
	\lstset{#1}
}
{}

% Python for external files
\newcommand\pythonexternal[2][]{{
		\pythonstyle
		\lstinputlisting[#1]{#2}}}

% Python for inline
\newcommand\pythoninline[1]{{\pythonstyle\lstinline!#1!}}

% Definition von Funktionen

% \paragraph mit Zeilenumbruch versehen ---------------------------
\makeatletter
\renewcommand\paragraph{\@startsection{paragraph}{4}{\z@}%
	{-3.25ex\@plus -1ex \@minus -.2ex}%
	{1.5ex \@plus .2ex}%
	{\normalfont\normalsize\bfseries}}
\makeatother
%------------------------------------------------------------------
