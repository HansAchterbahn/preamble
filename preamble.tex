% !TeX spellcheck = de_DE

%  *****************************************************
%  *  Name:     preamble.tex                           *
%  *  Editor:   Mario Hesse                            *
%  *  Version:  v0.1.2                                 *
%  *  Date:     24.10.2017                             *
%  *****************************************************


% Dokument / Seiten
% -----------------
\documentclass[12pt,a4paper,parskip=full+]{scrartcl}
\usepackage[utf8]{inputenc}				% Zeichencodierung
\usepackage[english,ngerman]{babel}		% Sprachpaket Englisch und Deutsch ausgewählt
\selectlanguage{ngerman}				% Sprache auf Deutsch gesetzt
\usepackage[T1]{fontenc}				% Umlaute bei \hyphenation verwenden können (Silbentrennung)
\usepackage[onehalfspacing]{setspace}	% Setzt das Dokument auf 1,5x Zeilenabstand (singlespacing/onehalfspacing/doublespacing)
\usepackage[headsepline]{scrpage2}		% Kopfzeile
\usepackage[defaultlines=4,all]{nowidow}				% Schusterjungen und Hurenkinder
\usepackage{							% standard Krempel
	amsmath,							% American Mathematical Society - Mathematik Basics (equation, equation*, align, align*, 
										% gather, gather*, flalign, flalign*, multline, multline*, alignat, alignat*, ...)
	amsfonts,							% Schriftarten für Mathematik und spezielle Symbole
	amsthm,								% erlaubt die Definiton von Theoremen
	amssymb}							% Mathematische Zeichen und Symbole



% Korrektur- und Dokumentatitonswerkzeuge
% ---------------------------------------
\usepackage{comment}					% Erlaubt das Verwenden der Umgebung \begin{comment}...\end{comment} -> Erstellen von Kommentaren über mehrere Zeilen
\usepackage[ngerman,
			disable,
%			textwidth=5cm,
			textsize=tiny,
			colorinlistoftodos,
			]{todonotes}				% Erlaubt das Verwenden von ToDo-Notitzen -> \todo{...} und Blindbildern -> \missingfigure{...}

\usepackage[ngerman]{translator}		% Erlaubt anderen Paketen Begriffe in Deutsche zu Übersetzen (Figure -> Abbildung, Table -> Tabelle, etc.)

\setlength{\headheight}{1.1\baselineskip}
\newcommand{\mario}[2][]{\todo[color=green,#1]{\textbf{mh:} #2}}
%\newcommand{\korrektor1}[1]{\todo[color=blue!40]{\textbf{ug:} #1}}
%\newcommand{\korrektor2}[2][]{\todo[color=purple!30,#1]{\textbf{wv:} #2}}


% Gleitumgebungen / Grafiken
% --------------------------
\usepackage{float}						% notwendig für Gleitubgebungen z.B. Figure
\usepackage{subfig}						% erlaubt das Verwenden von Subfigures
\usepackage{graphicx}					% erlaubt das Einbinden von Grafiken
\usepackage{pdfpages}					% ermöglicht das Einbinden von PDFs
\usepackage{epstopdf}					% einbinden von eps Bilddateien
\usepackage{afterpage}					% Im­ple­ments a com­mand that causes the com­mands spec­i­fied in its ar­gu­ment to be ex­panded af­ter the cur­rent page is out­put
\usepackage{tabularx}					% ermöglicht Tabbellen mit konkreter Breite festzulegen und erlaubt Zeilenumbruch in einer Tabelle

\newcolumntype{L}[1]{>{\raggedright\arraybackslash}p{#1}} 	% linksbündig mit Breitenangabe -> L{BREITE} statt p{BREITE} möglich
\newcolumntype{C}[1]{>{\centering\arraybackslash}p{#1}} 	% zentriert mit Breitenangabe -> C{BREITE} statt p{BREITE} möglich
\newcolumntype{R}[1]{>{\raggedleft\arraybackslash}p{#1}} 	% rechtsbündig mit Breitenangabe -> R{BREITE} statt p{BREITE} möglich
\newcolumntype{Y}{>{\centering\arraybackslash}X}			% X Spalten zentrieren
\newcolumntype{Z}{>{\raggedleft\arraybackslash}X}			% X Spalten zentrieren



\newcommand{\ltab}{\raggedright\arraybackslash} 	% Tabellenabschnitt linksbündig -> \ltab
\newcommand{\ctab}{\centering\arraybackslash} 		% Tabellenabschnitt zentriert -> \ctab 
\newcommand{\rtab}{\raggedleft\arraybackslash} 		% Tabellenabschnitt rechtsbündig -> \rtab


\usepackage{longtable}					% lange Tabellen verarbeiten
\usepackage{caption}					% ermöglicht die Beschriftung von Gleitobjekten
\usepackage{nameref}					% erlaubt Namensreferenzen
\usepackage{wrapfig}                    % Text umschließt Figure

\usepackage{forest}						% erlaubt einfache Verzeichnisstrukturen (tree)

% Diagramme
% ---------
\usepackage{tikz}						% Zeichentool
\usepackage{pgfplots, pgfplotstable}	% Diagramme zeichnen
\pgfplotsset{compat = newest}			% Einstellung für Diagramme
\usepackage[binary-units]{siunitx}      % SI-Einheiten
\usepgfplotslibrary{units}				% Einheiten in Diagrammen plotten


% Verzeichnisse
% -------------
\usepackage[square,authoryear]{natbib}	% Literatur zitieren
\bibliographystyle{IEEEtran}				% Literatur- und Zitatstyle | Alternativen: dinat, abbrvdin, alphadin, natdin, plain, abbrv, alpha, IEEEtran | Eigenen Style auf Konsole erzeugen: latex makebst |


% Inhaltsverzeichnis
\usepackage{makeidx}					% anlegen eines automatischen Inhaltsverzeichnis
\usepackage{tocstyle}					% Inhaltsverzeichnis
\usetocstyle{allwithdot} 				% Punkte im Inhaltsverzeichnis
\setcounter{tocdepth}{2}				% Inhaltsverzeichnis Ebenen definieren, zB bis zu Subsection


% Verlinkungen
% ------------
\usepackage[							% Verlinkung aller sections, ref , url, etc.
	pdfborderstyle={/S/U/W 1}				% Ändert den Rahmen der Links in Unterstriche
	% colorlinks=true,						% Schaltet Rahmen um die Links aus und ändert statt dessen die Schriftfarbe
	% urlcolor=blue,						% URL Links: Blau
	% citecolor=black,						% Zitate: Schwarz
	% linkcolor=black
	]{hyperref}		
\hypersetup{								% Erweiterte Einstellungen zu "hyperref" und erlaubt \autoref ()ist ähnlich wie \ref, schreibt aber Abbildung, Tabelle, usw. vor die verlinkte Gleitumgebung)
	bookmarks=true,							% Zeigt die Lesezeichenleiste an oder nicht.
	unicode=false,							% Ermöglicht die Verwendung von nicht-latin Schriftzeichen
	pdftoolbar=true,						% Setzt den Rahmen um einen Link. {0 0 0}erzeugt keinen Rahmen
	pdfmenubar=true,						% Zeigt die Acrobat-Toolbar an oder versteckt sie.
	pdffitwindow=false,						% Zeigt das Acrobat-Menü an oder versteckt es.
	pdfstartview={FitH},					% Verändert die Größe des Acrobat-Anzeigefensters, damit das Dokument hineinpasst.
	pdftitle={Titel},						% Definiert den Titel des Dokumentes, welcher in der Dokumenteninfo von Acrobat angezeigt wird.
	pdfauthor={Mario Hesse},						% Definiert den Namen des Autors für die Dokumenteninfo.
	pdfsubject={Beschreibung des PDF},					% Beschreibung des Dokument für die Dokumenteninfo.
	pdfcreator={Mario Hesse},				% Ersteller des Dokuments für die Dokumenteninfo.
	pdfproducer={Mario Hesse},							% Produzent des Dokuments für die Dokumenteninfo.
	pdfkeywords={Key1} {Key2} {Key3},		% Schlüsselwörter des Dokuments für die Dokumenteninfo (durch geschweifte Klammern voneinander getrennt; siehe unten).
	pdfnewwindow=true,						% Legt fest, dass Links in einem neuen Fenster geöffnet werden sollen.
	colorlinks=false,						% Legt fest, ob ein farbiger Rahmen um die Links gezogen werden soll oder ob die Schrift farbig sein soll.
	linkcolor=red,							% Farbe für die Links.
	citecolor=green,						% Farbe für die Quelllinks (bibliography; Quellenverzeichnis).
	filecolor=magenta,						% Farbe für Dateilinks.
	urlcolor=cyan							% Farbe für URL-Links (Web, Mail).
}

\renewcaptionname{ngerman}\sectionautorefname{Kapitel}				% Ändern von \autoref(section): Abschnitt 			-> Kapitel
\renewcaptionname{ngerman}\subsectionautorefname{Kapitel}			% Ändern von \autoref(section): Unterabschnitt 		-> Kapitel
\renewcaptionname{ngerman}\subsubsectionautorefname{Kapitel}		% Ändern von \autoref(section): Unterunterabschnitt	-> Kapitel
\newcommand{\subfigureautorefname}{\figureautorefname}		% Ändern von \autoref(subfloat): ~	-> tt

\usepackage{cleveref}						% \cref ist ähnlich wie \ref, schreibt aber Abb., Tab., usw. vor die verlinkte Gleitumgebung


% Glossar, Abkürzungsverzeichnis, Symbolverzeichnis, etc.
% -------------------------------------------------------
\usepackage[nonumberlist, 											% keine Seitenzahlen anzeigen
			acronym,      											% ein Abkürzungsverzeichnis erstellen
			toc,          											% Einträge im Inhaltsverzeichnis
			section      											% im Inhaltsverzeichnis auf section-Ebene erscheinen
			]{glossaries}										% Erlaubt Glossar, Abkürzungsverzeichnis, Symbolverzeichnis und mehr...
\newglossary[slg]{symbolslist}{syi}{syg}{Symbolverzeichnis}		% Ein eigenes Symbolverzeichnis erstellen
\renewcommand*{\glspostdescription}{}							% Den Punkt am Ende jeder Beschreibung deaktivieren
\newcommand{\glsit}[1]{\textit{\gls{#1}}}						% Neues Kommando \glos{} = \gls{} mit Link als kursivem Text
\newcommand{\glsplit}[1]{\textit{\glspl{#1}}}					% Neues Kommando \glospl{} = \glspl{} mit Link als kursivem Text
\makeglossaries													% Glossar-Befehle anschalten


% Formatierungen
% --------------
\usepackage{abstract}					% erlaubt des Verwenden der standartisierten Vorlage für Abstracts ( \begin{abstract}...\end{abstract} )
\addto\captionsngerman{%				% benennt den deutschen Titel des Abstracts von 'Zusammenfassung' um in 'Kurzfassung' 
	\renewcommand{\abstractname}{Kurzfassung}}
\usepackage{enumitem}					% ermöglicht das Einstellen von Abständen in itemize Umgebungen -> \setitemize{noitemsep,topsep=0pt,parsep=0pt,partopsep=0pt} (global) oder begin{itemize}[noitemsep,topsep=0pt,parsep=0pt,partopsep=0pt] (lokal) und erlaubt das Verwenden der {enumitem} Umgebung (itemize mit geringeren Abständen)
\setitemize{noitemsep,topsep=0pt,parsep=0pt,partopsep=0pt}
\usepackage{eurosym}					% €-Zeichen mit \euro darstellbar oder durch \EUR{betrag}
\usepackage{fdsymbol}					% Symboltabelle mit \hateq (Entspricht Symbol)
\usepackage{romannum}					% Römische Nummern schreiben mit '\romannum{integer}'
\newcommand{\Rom}[1]{\uppercase\expandafter{\romannumeral #1\relax}}		% Römische Nummern schreiben mit '\Rom{integer}' (nötig für Verwendung in \section{})
\setlength{\parindent}{0in} 			% Absatz, erste Zeile wird global NICHT eingerückt
%\usepackage{parskip}					% setzt einen Abstand nach einem \par Absatz
\usepackage{rotating}					% erlaubt das rotieren eines Elements
\newcommand\tabrotate[1]{\begin{turn}{90}\rlap{#1}\end{turn}}	% kurzbefehl für das Rotieren eines Elements

\usepackage{multirow}					% erlaubt in Tabellen das Zusammenfassen von Zellen in Reihe
\usepackage{booktabs}					% erlaubt mehrere unterschiedliche Linien in Tabellen (\toprule, \midrule, \bottomrule)
\usepackage{hhline}						% erlaubt individuelle Dicke bei Tabellenlinien

\usepackage{csquotes}
\newcommand{\gqt}[1]{\glqq #1\grqq{}}	% \gqt{...} (german quotation) ist eine Umgebung für deutsche Anführungszeichen | Alternativen: dirtytalk, csquotes, epigraph

% Custom colors
\usepackage{color}
\definecolor{deepblue}{rgb}{0,0,0.5}	% definiert ein tiefes Blau
\definecolor{deepred}{rgb}{0.6,0,0}		% definiert ein tiefes Rot
\definecolor{deepgreen}{rgb}{0,0.5,0}	% definiert ein tiefes Grün


\usepackage{wasysym}					% zusätzliche Symbole verwenden (Checkbox)
\usepackage{listings}					% ermölgicht Zeichengenaues Zitieren (für Programmcode) \verb|Zitat|


% Silbentrennung
%\expandafter\def\expandafter\UrlBreaks\expandafter{\UrlBreaks\do\a		% Erlaubt einen Umbruch in einer URL 
%	\do\b\do\c\do\d\do\e\do\f\do\g\do\h\do\i\do\j\do\k\do\l\do\m\do\n	% an allen angegebenen Stellen
%	\do\o\do\p\do\q\do\r\do\s\do\t\do\u\do\v\do\w\do\x\do\y\do\z\do\A	
%	\do\B\do\C\do\D\do\E\do\F\do\G\do\H\do\I\do\J\do\K\do\L\do\M\do\N	
%	\do\O\do\P\do\Q\do\R\do\S\do\T\do\U\do\V\do\W\do\X\do\Y\do\Z\do\1
%	\do\2\do\3\do\4\do\5\do\6\do\7\do\8\do\9\do\0\do\&}
\expandafter\def\expandafter\UrlBreaks\expandafter{\UrlBreaks
	\do.\do-\do:}													% nur an markanten Stellen

%\sloppy								% Nachlässige Silbentrennung, Blocksatzoptimiert
\hyphenation{OpenOCD To-po-lo-gie-in-for-ma-ti-onen Fer-ti-gungs-au-to-ma-ti-on Kom-mu-ni-ka-tions-ein-heit CANopen-Kom-mu-ni-ka-tions-ein-heit Atmos-phä-ren-druck ei-des-statt-li-ch UART USART El-ek-tro-den-an-ord-nung}					% Worte, die NICHT oder nur auf bestimmte Weise getrennt werden sollen

% \itemize mit Überschrift
\newenvironment{titlemize}[1]{
	\paragraph{#1}
	\begin{itemize}}
	{\end{itemize}}

% \paragraph mit Zeilenumbruch versehen 
% -begin-----------------------------------------------------------
\makeatletter
\renewcommand\paragraph{
	\@startsection{paragraph}{4}{\z@}
	{-3.25ex\@plus -1ex \@minus -.2ex}			% Raum vor \paragraph
	{1.5ex \@plus .2ex}							% Raum nach \paragraph
	{\normalfont\normalsize\bfseries}			% Text unter \paragraph
}
\makeatother
% -end-------------------------------------------------------------


% Python
% -begin----------------------------------------------------------
% Include Python-Programmcode with Syntax-Highlighting
% Default fixed font does not support bold face
\DeclareFixedFont{\ttb}{T1}{txtt}{bx}{n}{11} % for bold
\DeclareFixedFont{\ttm}{T1}{txtt}{m}{n}{11}  % for normal

% Python style for highlighting
\newcommand\pythonstyle{\lstset{
language=Python,
basicstyle=\ttm,
otherkeywords={self},             % Add keywords here
keywordstyle=\ttb\color{deepblue},
emph={MyClass,__init__},          % Custom highlighting
emphstyle=\ttb\color{deepred},    % Custom highlighting style
stringstyle=\color{deepgreen},
frame=tb,                         % Any extra options here
showstringspaces=false            % 
}}


\lstnewenvironment{python}[1][]		% Python environment
{
\pythonstyle
\lstset{#1}
}
{}

% Python for external files
\newcommand\pythonexternal[2][]{{
\pythonstyle
\lstinputlisting[#1]{#2}}}

% Python for inline
\newcommand\pythoninline[1]{{\pythonstyle\lstinline!#1!}}
% -end------------------------------------------------------------


% Citation aliases
%\defcitealias{2011IndustrialEthernetFacts}{Industrial Ethernet Facts [2011]}



% Fehler bekämpfen
\pdfoptionpdfminorversion=7
